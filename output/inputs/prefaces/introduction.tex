We are excited to welcome you to the Fifth Workshop on Language Technology for Equality, Diversity, Inclusion (LT-EDI-2025), the 5th Conference on Language, Data and Knowledge (LDK). This year, the workshop will be held in a hybrid format (both online and Workshops will take place at Palazzo del Mediterraneo on 9th September 2025, while the main venue for the conference will be Palazzo Corigliano, on 10th - 11th September 2025, located in the Naples, Italy. With the rapid advancement of technology, digital communication has become a central part of daily life. While many globally dominant languages have successfully transitioned into the digital era, numerous regional and low-resource languages continue to face significant technological challenges. Equality, Diversity and Inclusion (EDI) is an important agenda across every field throughout the world. Language as a major part of communication should be inclusive and treat everyone with equality. Today’s large internet community uses language technology (LT) and has a direct impact on people across the globe. EDI is crucial to ensure everyone is valued and included, so it is necessary to build LT that serves this purpose. Recent results have shown that big data and deep learning are entrenching existing biases and that some algorithms are even naturally biased due to problems such as ‘regression to the mode’. Our focus is on creating LT that will be more inclusive of gender, racial, sexual orientation, persons with disability. The workshop will focus on creating speech and language technology to address EDI not only in English, but also in less resourced languages. The workshop received a total of 40 active submissions. Reviewer recruitment was highly effective, with 232 out of 249 invited reviewers accepting the invitation. Of the 270 assigned reviews, 117 were completed, resulting in a review submission rate of 43.33\%. Additionally, 41.67\% of reviewers (100 out of 240) completed all their assigned reviews. A majority of submissions (65\%, or 26 out of 40) received at least three reviews, ensuring a robust evaluation process. Decisions were finalized for all submissions (100\%), leading to an acceptance rate of 95\% (38 papers). This included 6 papers (15\%) accepted for oral presentations and 32 papers (80\%) accepted for poster presentations. Only 2 submissions (5\%) were rejected. There were no withdrawn submissions, and only one paper was desk rejected. These metrics reflect a thorough and inclusive review process, driven by active reviewer participation and a strong commitment to quality.